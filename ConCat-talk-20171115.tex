\documentclass[10pt]{beamer}

\usetheme{metropolis}
\usepackage{appendixnumberbeamer}

\usepackage{booktabs}
\usepackage[scale=2]{ccicons}

\usepackage{pgfplots}
\usepgfplotslibrary{dateplot}

\usepackage{xspace}
\newcommand{\MDL}{\textbf{\textsc{mdl}}\xspace}

\usepackage{graphicx}
\usepackage{tikz-cd}

\newcommand{\Cat}[1]{\ensuremath{\underline{\mathbf{#1}}}}
\newcommand{\Obj}[1]{\ensuremath{\mathrm{Obj}(\Cat{#1})}}
\newcommand{\Hom}[3]{\ensuremath{\mathrm{Hom}_{\Cat{#1}}(#2,#3)}}
\newcommand{\Domain}[1]{\ensuremath{\mathbf{Dom}(#1)}}
\newcommand{\Codomain}[1]{\ensuremath{\mathbf{Cod}(#1)}}
\newcommand{\CompCat}[1]{\Cat{\overline{#1}}}
\newcommand{\Comm}[1]{\mathrm{Kom}_{\Cat{#1}}}
\newcommand{\Com}[3]{#3^{#2}}
\newcommand{\SCat}[1]{\Cat{Symms(#1)}}
\newcommand{\FSCat}[1]{\Cat{FSymm(#1)}}
\newcommand{\FkCat}[1]{\Cat{FkSymm(#1)}}
\newcommand{\PkCat}[1]{\Cat{PkSymm(#1)}}
\newcommand{\IrPaths}[1]{\Cat{IrPaths(#1)}}
\newcommand{\FProj}[2]{\ensuremath{W_{\Cat{#1,#2}}}}
\newcommand{\FuncIm}[1]{\mathrm{Im}(#1)}
\newcommand{\term}[1]{\emph{#1}}
\newcommand{\strong}[1]{\textbf{#1}}

%% BEGIN Conal Elliott paper macros
\newcommand{\program}[1]{\texttt{#1}}
\newcommand{\id}{\mathtt{id}}
\newcommand{\apply}{\mathtt{apply}}
\newcommand{\curry}{\mathtt{curry}}
\newcommand{\uncurry}{\mathtt{uncurry}}
\newcommand{\exl}{\mathtt{exl}}
\newcommand{\exr}{\mathtt{exr}}
\newcommand{\termin}{\mathtt{1}}
\newcommand{\termarr}{\mathtt{it}}
\newcommand{\unitarrow}{\ensuremath{\mathtt{unitarrow}}}
\newcommand{\lamf}[2]{\ensuremath{\lambda\, #1 \mapsto #2}}
\newcommand{\ccc}{\ensuremath{\mathtt{ccc}}}
\newcommand{\lamtoccc}[1]{\ensuremath{\ccc (#1)}}
\newcommand{\delprod}[2]{\ensuremath{#1\,\Delta\,#2}}
\newcommand{\const}{\ensuremath{\mathtt{const}}}
\newcommand{\constFun}{\ensuremath{\mathtt{constFun}}}
\newcommand{\credit}[1]{{\footnotesize #1}}
\newcommand{\Dist}[1]{\ensuremath{\mathtt{Dist}\,#1}}
%% END Conal Elliott paper macros

\newcommand{\eqnlabel}[1]{\label{eq:#1}}
\newcommand{\eqnref}[1]{(\ref{eq:#1})}
\newcommand{\deflabel}[1]{\label{def:#1}}
\newcommand{\defref}[1]{definition \ref{def:#1}}
\newcommand{\thlabel}[1]{\label{th:#1}}
\newcommand{\thref}[1]{theorem \ref{th:#1}}

% \newcommand{\pullback}{\arrow[d,]}
% \newcommand{\pushout}{\arrow[d,\text{\pigpenfont J}]}

% \newtheorem{threm}{Theorem}[section]
% \newtheorem{lemma}[threm]{Lemma}
\theoremstyle{definition}
% \newtheorem{corollary}[theorem]{Corollary}
% \newtheorem{definition}[theorem]{Definition}
% \newtheorem{example}[theorem]{Example}
\newtheorem{xca}[theorem]{Exercise}

\theoremstyle{remark}
\newtheorem{remark}[theorem]{Remark}

\numberwithin{equation}{section}

%    Absolute value notation
\newcommand{\abs}[1]{\lvert#1\rvert}

%    Blank box placeholder for figures (to avoid requiring any
%    particular graphics capabilities for printing this document).
\newcommand{\blankbox}[2]{%
  \parbox{\columnwidth}{\centering
%    Set fboxsep to 0 so that the actual size of the box will match the
%    given measurements more closely.
    \setlength{\fboxsep}{0pt}%
    \fbox{\raisebox{0pt}[#2]{\hspace{#1}}}%
  }%
}


\title{Compiling to Categories}
\subtitle{Mathematically-principled program transformation}
% \subtitle{Our attempt to explain what Conal Elliott is up to}
\date{\today}
\author{T. Mark Ellison and Siva Kalyan}
\institute{Australian National University}
% \titlegraphic{\hfill\includegraphics[height=1.5cm]{logo.pdf}}

\usepackage{listings}
\usepackage{color}
\definecolor{identifierColor}{rgb}{0.65,0.16,0.16}
\lstset{
  frame=none,
  xleftmargin=2pt,
  stepnumber=1,
  numbers=left,
  numbersep=5pt,
  numberstyle=\ttfamily\tiny\color[gray]{0.3},
  belowcaptionskip=\bigskipamount,
  captionpos=b,
  escapeinside={*'}{'*},
  language=haskell,
  tabsize=2,
  emphstyle={\bf},
  commentstyle=\it\color{brown},
  stringstyle=\mdseries\rmfamily\color{green},
  showspaces=false,
  keywordstyle=\bfseries\rmfamily\color{red},
  columns=flexible,
  basicstyle=\small\sffamily\color{blue},
  showstringspaces=false,
  morecomment=[l]\%,
}

\begin{document}

%% SEE https://wiki.haskell.org/wikiupload/8/85/TMR-Issue13.pdf#page=73

\maketitle

\begin{frame}{Table of contents}
  \setbeamertemplate{section in toc}[sections numbered]
  \tableofcontents[hideallsubsections]
\end{frame}


\begin{frame}[fragile]
  \frametitle{Haskell and Category Theory}

  \begin{center}
    \begin{tabular}{lr}
    \toprule
    Haskell & Category Theory \\
    \midrule
    \textbf{Category} & \textbf{Category} \\
    \textbf{Type} & \textbf{Object} \\
    \textbf{Function} & \textbf{Morphism} \\
    \textbf{\Cat{Hask}} & \textbf{\Cat{Set}} \\
    \textbf{...} & \textbf{Terminal Objects} \\
    \textbf{Tuple} & \textbf{Product} \\
    \textbf{Currying, Function Application} & \textbf{Cartesian Closure} \\
    \bottomrule
  \end{tabular}
  \end{center}

\end{frame}

\section{Closed Cartesian Categories in Category Theory}

\subsection{Basics}

\begin{frame}[fragile]
  \frametitle{Categories}

  A \emph{category} $\Cat{C}$ consists of
  \begin{enumerate}
  \item a class $\Obj{C}$ of \emph{objects}, and
  \item for each pair of objects $A,B \in \Obj{C}$, a set $\Hom{C}{A}{B}$
    of \emph{arrows} (or \emph{morphisms}) from $A$ to $B$, known as
    a \emph{hom-set}.
  \end{enumerate}\vspace{-1.5\baselineskip}

  \[
    \begin{tikzcd}
      A \arrow[shift left=2]{r}{\Hom{C}{A}{B}} \arrow{r} \arrow[shift right=2]{r} & B
    \end{tikzcd}
  \]

  Many familiar parts of Haskell form a category \Cat{Hask}:
  objects are \emph{types} (\lstinline{Int}, \lstinline{Char}, etc.),
  and arrows are \emph{functions} between types (e.g.\ \lstinline{ord :: Int -> Char}).
\end{frame}

\begin{frame}[fragile]
  \frametitle{Category Laws}
  In a category $\Cat{C}$:
  \begin{enumerate}
  \item Given arrows $f\colon A \rightarrow B$ and $g\colon B \rightarrow C$ in
    $\Cat{C}$, the \emph{composition} $g \circ f \colon A \rightarrow C$ (= \lstinline{g.f}) is also in
    $\Cat{C}$.
  \item Given arrows $f\colon A \rightarrow B$, $g\colon B \rightarrow C$ and $h\colon C \rightarrow D$, $(h \circ
    g) \circ f = h \circ (g \circ f) = h \circ g \circ f$:\vspace{-0.5\baselineskip}
    \[
      \begin{tikzcd}
        A \arrow{r}{f} \arrow[sloped,near end,swap,dashed]{dr}{g \circ f} \arrow[bend
        right=60,sloped,swap,loosely dashed]{drr}{h \circ g \circ f} \arrow[bend
        left=60,sloped,loosely dashed]{drr}{h
          \circ g \circ f} & B \arrow{d}{g}
        \arrow[sloped,near start,dashed]{dr}{h \circ g} & \\
          & C \arrow[swap]{r}{h} & D
      \end{tikzcd}\rlap{.}
    \]
  \item Every object $A \in \Obj{C}$ is associated with an \emph{identity arrow}
    $1_A \colon A \rightarrow A$ (= \lstinline{id}). Given any arrow $f\colon A \rightarrow B$, we have
    \[
      \begin{tikzcd}
        A \arrow[densely dotted]{r}{1_A} \arrow[bend right=30,swap]{rr}{f} & A
        \arrow["f" description]{r}
        \arrow[bend left=30]{rr}{f} & B \arrow[swap,densely dotted]{r}{1_B} & B
      \end{tikzcd}\rlap{.}
    \]
  \end{enumerate}
\end{frame}

\begin{frame}[fragile]
  \frametitle{Examples}

  \begin{center}
    \begin{tabular}{r l l l l}\toprule
    & $\Cat{Set}$ & $\Cat{Hask}$ & $\Cat{POrd}$ & $\Cat{Cat}$ \\\midrule
    \textbf{Objects} & sets & types & items & small cats \\
    \textbf{Morphisms} & functions & functions & $a \leq b$ & functors \\
    \textbf{Composition} & $f \circ g$ & \lstinline!f.g! & transitivity & $F \circ G$ \\
    \textbf{Identity} & $1_A$ & {\lstinline!id!} & $a = a$ & $1_{\Cat{C}}$ \\\bottomrule
  \end{tabular}
  \end{center}

  Not everything in Haskell can be in $\Cat{Hask}$ if we want it to be a category. Every type in the language contains a \lstinline{Bottom} ($\bot$) or \lstinline{undefined} value, but these 'values' cause mayhem with the category laws (in particular the \textbf{Identity} constraint). So when we talk about \Cat{Hask} we'll be talking about vanilla \Cat{Hask} without these abnormal values.
  {\small (Haskell wiki page on \Cat{Hask}.)}
\end{frame}

\begin{frame}[fragile]{Category Theory: Terminal Objects}

  A \emph{terminal object} is a type $1$ (a.k.a.\ $T$) in $\Obj{C}$, such that there is only a single mapping from any other type $A$ onto that type:

  \[
    \forall A\in \Obj{C}, \left| \Hom{C}{A}{1} \right| = 1.
  \]

  \[
    \begin{tikzcd}[row sep=tiny]
      A \arrow[near start]{dr}{\exists!} & \\
      B \arrow["\exists!" description]{r} & 1\\
      C \arrow[swap,near start]{ur}{\exists!} & 
    \end{tikzcd}
  \]
  
  % Terminal objects in categories:
  % \begin{description}
  %   \item[\Cat{Set}] the singleton set
  %   \item[\Cat{Grp}] the singleton group
  %   \item[\Cat{Top}] the singleton topological space
  %   \item[\Cat{1\downarrow Set}] the singleton pointed set
  %   \item[\Cat{POrd}] the maximum element (if there is one)
  % \end{description}

  In \Cat{Hask}:
  \begin{lstlisting}[frame=single]
    () -- the type corresponding to 1, containing only itself
    terminalMap :: t -> ()
    terminalMap _ = ()
  \end{lstlisting}

\end{frame}

\begin{frame}[fragile]
  \frametitle{Examples}

  \begin{center}
    \begin{tabular}{r l l l l}\toprule
    & $\Cat{Set}$ & $\Cat{Hask}$ & $\Cat{POrd}$ & $\Cat{Cat}$ \\\midrule
    \textbf{Objects} & sets & types & items & small cats \\
    \textbf{Morphisms} & functions & functions & $a \leq b$ & functors \\
    \textbf{Composition} & $f \circ g$ & \lstinline!f.g! & transitivity & $F \circ G$ \\
    \textbf{Identity} & $1_A$ & {\lstinline!id!} & $a = a$ & $1_{\Cat{C}}$ \\
%%    \textbf{Initial obj\rlap{.}} & $\emptyset$ & \texttt{Void} & lwr bnd & $\Cat{0}$ \\
      \textbf{Terminal obj\rlap{.}} & $\{*\}$ & \lstinline!()! & upper bound & $\Cat{1}$ \\\bottomrule
  \end{tabular}
  \end{center}
  
\end{frame}

\begin{frame}[fragile]{Products}

  Given objects $A$, $B$ in $\Cat{C}$ there may be a \emph{(pairwise) product}
  $A\sqcap B \in \Obj{C}$ and \emph{projection arrows} $\pi_A \colon A \sqcap B \rightarrow A$ and
  $\pi_B \colon A \sqcap B \rightarrow B$ such that for any object $X$ in the same category and arrows
  $a \colon X \rightarrow  A$ and $b \colon X \rightarrow B$
  there is a \emph{unique} arrow $x : X \rightarrow A \sqcap B$ such that $a = \pi_A \circ x$ and $b = \pi_B \circ x$:

  \[
  \begin{tikzcd}
    A & A\sqcap B \arrow[swap]{l}{\pi_{A}} \arrow{r}{\pi_{B}} & B \\
    & X \arrow[dotted]{u}{\exists! x} \arrow{ul}{a} \arrow[swap]{ur}{b} & 
  \end{tikzcd}\rlap{.}
  \]

  In other words: Given a particular way of mapping $X$ to $A$ and to $B$, there's only \emph{one} way of mapping $X$ to $A \sqcap B$ such that everything's consistent.

\end{frame}

\begin{frame}[fragile]{Products}

  Alternatively, the triplet $\langle {A \sqcap B, \pi_A, \pi_B} \rangle$ is a \emph{terminal object}
  in the category whose objects are diagrams of the form
  \[
    \begin{tikzcd}
      A & C \arrow{l} \arrow{r} & B
    \end{tikzcd}\rlap{,}
  \]
  and whose arrows are (commutative) diagrams of the form
  \[
    \begin{tikzcd}[row sep=tiny]
       & C \arrow[bend right=15]{dl} \arrow[bend left=15]{dr} & \\
      A & & B \\
       & D \arrow[bend left=15]{ul} \arrow[swap]{uu}{f} \arrow[bend right=15]{ur} & 
    \end{tikzcd}\rlap{.}
  \]

\end{frame}

\begin{frame}[fragile]
  \frametitle{Products in Haskell}
  \begin{lstlisting}[frame=single,mathescape=true]
    (a,b) -- the type containing pairs from types a and b ($A \sqcap B$)
    fst :: (a,b) -> a -- the projection function $\pi_A$
    fst (x,y) = x
    snd :: (a,b) -> b -- the projection function $\pi_B$
    snd (x,y) = y
    factorThroughProd :: (c -> a) -> (c -> b) -> (c -> (a,b))
    factorThroughProd f g = \ x -> (f x,g x)
  \end{lstlisting}

  It should be obvious that\\
  \lstinline{fst.(factorThroughProd f g) = f}, and\\
  \lstinline{snd.(factorThroughProd f g) = g}.
  
\end{frame}

\begin{frame}[fragile]
  \frametitle{Examples}

  \begin{center}
    \begin{tabular}{r l l l l}\toprule
    & $\Cat{Set}$ & $\Cat{Hask}$ & $\Cat{POrd}$ & $\Cat{Cat}$ \\\midrule
    \textbf{Objects} & sets & types & items & small cats \\
    \textbf{Morphisms} & functions & functions & $a \leq b$ & functors \\
    \textbf{Composition} & $f \circ g$ & \lstinline!f.g! & transitivity & $F \circ G$ \\
    \textbf{Identity} & $1_A$ & {\lstinline!id!} & $a = a$ & $1_{\Cat{C}}$ \\
    % \textbf{Initial obj\rlap{.}} & $\emptyset$ & \lstinline!Void! & lower bound & $\Cat{0}$ \\
    \textbf{Terminal obj\rlap{.}} & $\{*\}$ & \lstinline!()! & upper bound & $\Cat{1}$ \\
    \textbf{Product} & $A \times B$ & \lstinline!(a,b)! & $\min(a,b)$ & $\Cat{C} \times \Cat{D}$ \\
    % \textbf{Coproduct} & $A \uplus B$ & \lstinline!Either a b! & $\max(a,b)$ & $\Cat{C} \sqcup \Cat{B}$ \\
    \bottomrule
  \end{tabular}
  \end{center}
\end{frame}

\begin{frame}[fragile]
  \frametitle{Exponential Objects}

  Given objects $A$ and $B$ in $\Cat{C}$, an \emph{exponential object} $B^A$
  (also written $[A \rightarrow B]$) is an object with an arrow $\mathrm{eval}_B^A$
  such that for any $C$ and any arrow $f\colon C \sqcap A \rightarrow B$,
  \[
  \begin{tikzcd}
    C \sqcap A \arrow[swap,densely dotted]{d}{\exists!} \arrow{dr}{f} & \\
    B^A \sqcap A \arrow[swap]{r}{\mathrm{eval}_B^A} & B
    \end{tikzcd}\rlap{.}
  \]
  
  Alternatively, the pair $\langle {B^A,\mathrm{eval}_B^A} \rangle$ constitutes a terminal
  object in the category whose objects are diagrams of the form
  \[
  \begin{tikzcd}
    C \sqcap A \arrow{r} & B
  \end{tikzcd}\rlap{,}
  \]
  and whose arrows are commutative diagrams of the form
  \[
  \begin{tikzcd}[row sep=tiny]
    D \sqcap A \arrow{dd} \arrow[bend left=15]{dr} & \\
    & B \\
    C \sqcap A \arrow[bend right=15]{ur} &
  \end{tikzcd}\rlap{.}
  \]
\end{frame}

\begin{frame}[fragile,fragile]
  \frametitle{Exponential Objects in Haskell}

  In $\Cat{Hask}$, the exponential object of two types \lstinline{a} and
  \lstinline{b} is the \emph{function type} \lstinline{(a -> b)} (it's akin to
  the \emph{hom-set} of \lstinline{a} and \lstinline{b}). Let's see
  how this satisfies the above definition.

  \begin{lstlisting}[frame=single]
    eval :: ((a -> b),a) -> b
    eval (f,x) = f x
    factoredArrow :: ((c,a) -> b) -> ((c,a) -> ((a -> b),a))
    factoredArrow f = \ (y,x) -> ((\ x' -> f(y,x')),x)
  \end{lstlisting}
  {\footnotesize{(Spot the currying!)}}

  It can be proven that \lstinline{eval . (factoredArrow f) = f} --- and that
  \lstinline{factoredArrow} is the \emph{only} arrow for which this is true.

\end{frame}

\begin{frame}[fragile]{Cartesian-Closed Categories (CCC)}

  There is a terminal object $1$.

  There are binary products $\sqcap$ (and hence all finite products).

  For any two objects $A$ and $B$, there is an exponential object $B^A$.

  Examples:
  \begin{description}
    \item[\Cat{Set}] the singleton set, pairs, sets of functions
    \item[\Cat{Hask}] \lstinline{()}, \lstinline{(a,b)}, \lstinline{a -> b}
  \end{description}

  There are more examples, but they're pretty complicated.

\end{frame}

\section{Conal Elliott: Compiling to Categories} %% Mark

\begin{frame}[fragile]{Compiling to Categories}
  So far, we have introduced concepts from standard category theory, with a bit of Haskell flavour.

  It is well-known that Haskell (or a near-complete subset of it) has category-theoretic semantics (e.g. our last talk), given in terms of a single category $\Cat{Hask}$.

  Elliott's (2017) paper \emph{Compiling to Categories} (hereafter C2C) shows that Category Theory can not only provide semantics, but a range of compile-to domains to which \emph{the same code} can be compiled.

  Single most exciting paper in the interpretation of programming languages.
\end{frame}

\begin{frame}[fragile]{Compiling to Categories}
  \begin{center}
    \includegraphics[width=1.0\textwidth]{compiling-to-categories-headpic.png}
  \end{center}
\end{frame}

\begin{frame}[fragile]{Compiling to Categories}
  How it works (black box):
  \begin{itemize}
    \item you specify (using Haskell classes) the application category
    \item then Haskell code is compiled to constructions in that category
    \item while the constructions reflect the structure of your program, they do not simply implement it.
  \end{itemize}
\end{frame}

\begin{frame}[fragile]{Compiling to Categories}
  Why this is exciting:
  
  By choosing different CCCs, you can do these things (CCC names not the same as in C2C, but you can work it out):
  \begin{description}
  \item[free CCC] pretty-printing, syntax-highlighting, or proving correctness
  \item[intervals] verification
  \item[delta] partial memoisation
  \item[hardware] translate software into hardware
  \item[linear spaces] linear approximations to complex numeric functions
  \item[differentials] differentiate any haskell numeric function - automatically
  \end{description}
\end{frame}

\begin{frame}[fragile]{Compiling to Categories: Overview}
  How it works (under the hood):
  \begin{itemize}
  \item compile Haskell $\rightarrow$ $\lambda$-expressions (grab intermediate output from GHC)
  \item $\lambda$-expressions $\rightarrow$ CCC-constructions
  \item CCC-constructions applied in category of choice
  \item output result
  \end{itemize}
\end{frame}

\begin{frame}[fragile]{Declaring CCCs}
  Example: we want to compile numeric expressions/functions into something that tells us about the bounds on outputs (minimum possible output and maximum possible output).

  This cannot be achieved with a black-box 2nd-order function, except by enumerating possible inputs.

  But can be achieved by compilation to categories.
\end{frame}

\begin{frame}[fragile]{Declaring CCCs}
  First we define the \lstinline{type family} of intervals. Here \lstinline{:*} is a pairing operator.
  \begin{lstlisting}[frame=single]
    type family Iv a
    type instance Iv ()     = ()
    type instance Iv Float  = Float  :* Float
    type instance Iv Double = Double :* Double
    type instance Iv Int    = Int    :* Int
  \end{lstlisting}
\end{frame}

\begin{frame}[fragile]{Declaring CCCs}
  Now we define our category. First the data type \lstinline{IF} which contains our morphisms.
  \begin{lstlisting}[frame=single]
    data IF a b = IF { unIF :: Iv a -> Iv b }
  \end{lstlisting}
\end{frame}

\begin{frame}[fragile]{Declaring CCCs}
  I'm using \lstinline{pack0}, \lstinline{pack1}, \lstinline{pack2} to map functions of 0-, 1- and 2-arguments in $\Cat{Hask}$ into the new category. Elliott's code uses \lstinline{pack}, \lstinline{inNew} and \lstinline{inNew2}.
  \begin{lstlisting}[frame=single]
instance Category IF where
  id = pack0 id
  (.) = pack2 (.)
  \end{lstlisting}
\end{frame}

\begin{frame}[fragile]{Declaring CCCs}
  \begin{lstlisting}[frame=single]
instance ProductCat IF where
  exl = pack0 exl
  exr = pack0 exr
  (&&&) = pack2 (&&&)
  \end{lstlisting}
\end{frame}

\begin{frame}[fragile]{Declaring CCCs}
  \begin{lstlisting}[frame=single]
instance ClosedCat IF where
  apply = pack0 apply
  curry = pack1 curry
  uncurry = pack1 uncurry
  \end{lstlisting}
\end{frame}

\begin{frame}[fragile]{Declaring CCCs}
  \begin{lstlisting}[frame=single]
instance Iv b ~ (b :* b) => ConstCat IF b where
  const b = pack0 (const (b,b))
  unitArrow b = pack0 (unitArrow (b,b))
  \end{lstlisting}
\end{frame}


\begin{frame}[fragile]{Declaring CCCs}
  Now define how some atomic Haskell functions map into the CCC.
  \begin{lstlisting}[frame=single]
instance (Iv a ~ (a :* a), Num a, Ord a) => NumCat IF a where
  negateC = pack0 (\ (al,ah) -> (-ah, -al))
  addC = pack0 (\ ((al,ah),(bl,bh)) -> (al+bl,ah+bh))
  subC = addC . second negateC
  mulC = pack0 (\ ((al,ah),(bl,bh)) ->
               let cs = ((al*bl,al*bh),(ah*bl,ah*bh)) in
                 (min4 cs, max4 cs))
  \end{lstlisting}
\end{frame}

\begin{frame}[fragile]{Compiling to $\lambda$-expressions}
  \credit{Credit: \url{https://stackoverflow.com/questions/27635111}.}
  \begin{itemize}
    \item use the \lstinline{GHC} module
    \item functions \lstinline{compileToCoreModule} or \lstinline{compileToCoreSimplified} to compile a file
    \item the code has been reproduced as \lstinline{processor.hs} in the repository with today's talk. You need to compile it with
      \begin{lstlisting}
        $ ghc -package ghc -package ghc-paths processor.hs
      \end{lstlisting}
  \end{itemize}
\end{frame}

\begin{frame}[fragile]{Haskell to $\lambda$-Calculus}
  \begin{lstlisting}[frame=single]
  example :: Int -> Int -> Int
  example x y = x + y
  \end{lstlisting}

  \begin{lstlisting}[frame=single]
  example = \ (x :: Int) (y :: Int) -> + @ Int $fNumInt x y
  \end{lstlisting}
\end{frame}

\begin{frame}[fragile]{Haskell to $\lambda$-Calculus}
  \begin{lstlisting}[frame=single]
  example :: Int -> Int -> Int
  example x y = x + y
  \end{lstlisting}

  \begin{lstlisting}[frame=single]
  example = \ (x :: Int) (y :: Int) -> + @ Int $fNumInt x y
  \end{lstlisting}
\end{frame}


\begin{frame}[fragile]{From $\lambda$-Calculus to CCCs}
  The mapping operation is implemented as a pseudo-function $\ccc$.

  Each transformation either reduces the size of the body of the $\lambda$-expression,
  or eliminates a $\lambda$.

  Consequently, the transformation process must terminate.
\end{frame}

\begin{frame}[fragile]{Category Definition}
  \[
    \begin{tikzcd}
      A \arrow[shift left=2]{r}{\Hom{C}{A}{B}} \arrow{r} \arrow[shift right=2]{r} & B
    \end{tikzcd}
  \]
  \[
  \begin{tikzcd}
    A \arrow{r}{f} \arrow[sloped,near end,swap,dashed]{dr}{g \circ f} \arrow[bend
      right=60,sloped,swap,loosely dashed]{drr}{h \circ g \circ f} \arrow[bend
      left=60,sloped,loosely dashed]{drr}{h
      \circ g \circ f} & B \arrow{d}{g}
    \arrow[sloped,near start,dashed]{dr}{h \circ g} & \\
    & C \arrow[swap]{r}{h} & D
  \end{tikzcd}\rlap{.}
  \]
  \[
  \begin{tikzcd}
    A \arrow[densely dotted]{r}{1_A} \arrow[bend right=30,swap]{rr}{f} & A
    \arrow["f" description]{r}
    \arrow[bend left=30]{rr}{f} & B \arrow[swap,densely dotted]{r}{1_B} & B
  \end{tikzcd}\rlap{.}
  \]
\end{frame}

\begin{frame}[fragile]{Category Definition}
  \begin{itemize}
  \item composition $g \circ f = \lamf{x}{g(f(x))}$.
  \item identity $\id = \lamf{x}{x} $,
  \end{itemize}
  
  Laws:\\
  \begin{itemize}
  \item $\id \circ f \equiv f \circ \id \equiv f$
  \item $h \circ (g \circ f) \equiv (h \circ g) \circ f$
  \end{itemize}
\end{frame}

\begin{frame}[fragile]{Expression Body is a Single Variable}
  $\lamtoccc{\lamf{x}{x}} = \id$
\end{frame}

\begin{frame}[fragile]{A Terminal Object}
  \begin{center}
    \[
    \begin{tikzcd}[row sep=tiny]
      A \arrow[near start]{dr}{\exists!} & \\
      B \arrow["\exists!" description]{r} & 1\\
      C \arrow[swap,near start]{ur}{\exists!} & 
    \end{tikzcd}
    \]
  \end{center}
\end{frame}

\begin{frame}[fragile]{A Terminal Object}
  \begin{itemize}
  \item terminal $\termin$ is the terminal object in the category,
  \item terminal arrow $\termarr = \lamf{a}{()}$.
  \item unitarrow $\unitarrow\, b = \lamf{()}{b}$.
  \item constants $\const\, b = (\unitarrow\, b) \circ \termarr$
  \end{itemize}
  
  Laws:\\
  \begin{itemize}
  \item $\termarr \circ f \equiv \termarr$
  \end{itemize}
\end{frame}

\begin{frame}[fragile]{Simple Constants}
  $\lamtoccc{\lamf{x}{c}} = \const\, c$
\end{frame}

\begin{frame}[fragile]{The Product}
  \[
    \begin{tikzcd}[row sep=tiny]
       & C \arrow[bend right=15]{dl} \arrow[bend left=15]{dr} & \\
      A & & B \\
       & D \arrow[bend left=15]{ul} \arrow[swap]{uu}{f} \arrow[bend right=15]{ur} & 
    \end{tikzcd}\rlap{.}
  \]
\end{frame}

\begin{frame}[fragile]{The Product}
  \begin{itemize}
  \item fork $\delprod{f}{g} = \lamf{x}{(f\,x,g\,x)}$,
  \item extract-left $\exl = \lamf{(a,b)}{a}$,
  \item extract-right $\exr = \lamf{(a,b)}{b}$.
  \end{itemize}
  
  Laws:\\
  \begin{itemize}
  \item $\exl\circ(\delprod{f}{g}) \equiv f$
  \item $\exr\circ(\delprod{f}{g}) \equiv g$
  \item $\delprod{\exl\circ h}{\exr\circ h} \equiv h$
  \end{itemize}
\end{frame}

\begin{frame}[fragile]{Exponential Objects}
  \begin{center}
    \[
    \begin{tikzcd}
      C \sqcap A \arrow[swap,densely dotted]{d}{\exists!} \arrow{dr}{f} & \\
      B^A \sqcap A \arrow[swap]{r}{\mathrm{eval}_B^A} & B
    \end{tikzcd}\rlap{.}
    \]
  \end{center}
\end{frame}

\begin{frame}[fragile]{Exponential Objects}
  \begin{itemize}
  \item apply or eval $\apply\, (f,x) = f\,x$
  \item $\curry\, f = \lamf{a\,b}{f\,(a,b)}$
  \item $\uncurry\, f = \lamf{(a,b)}{f\,a\,b}$
  \item constant functions $\constFun\, f = \curry (f \circ exr) = \lamf{x}{f}$ ignores $x$, returns a function
  \end{itemize}
  
  Laws:\\
  \begin{itemize}
  \item $\uncurry\,(\curry\,f) \equiv f$
  \item $\curry\,(\uncurry\,f) \equiv f$
  \item $\apply \circ (\delprod{\curry\,f\circ\exl}{\exr}) \equiv f$
  \end{itemize}
\end{frame}

\begin{frame}[fragile]{Expression Body is an Application}
  \begin{description}
    \item[Expression body is an application]
      $$\lamtoccc{\lamf{x}{U\,V}} = \apply \circ (\delprod{\lamtoccc{\lamf{x}{U}}}{\lamtoccc{\lamf{x}{V}}})$$
    \item[Lambda abstraction]
      $$\lamtoccc{\lamf{x}{\lamf{y}{U}}} = \curry\, \lamtoccc{\lamf{(x,y)}{U}}$$
    \item[Constant functions]
      $$\lamtoccc{\lamf{x}{f}} = \constFun\, \lamtoccc{f}$$
      $f$ may need to be \emph{Curried} to reduce its argument dimensionality.
  \end{description}
\end{frame}

\section{Examples} %% Mark

\begin{frame}[fragile]{Syntactic Analysis}
  The simplest application is just to build a tree structure of the functions applying in the CCC.

  Each function just records a label (the same as its name) on a tree node, and then builds substrees from any arguments.
\end{frame}

\begin{frame}[fragile]{Syntactic Analysis}
  \begin{lstlisting}
appt :: String -> [DocTree] -> DocTree
appt = Node . const . text
-- appt s ts = Node (const (text s)) ts
  \end{lstlisting}
\end{frame}

\begin{frame}[fragile]{Syntactic Analysis}
  \begin{lstlisting}
atom :: Pretty a => a -> Syn a b
atom a = Syn (Node (ppretty a) [])

app0 :: String -> Syn a b
app0 s = Syn (appt s [])

app1 :: String -> Syn a b -> Syn c d
app1 s (Syn p) = Syn (appt s [p])

app2 :: String -> Syn a b -> Syn c d -> Syn e f
app2 s (Syn p) (Syn q) = Syn (appt s [p,q])
  \end{lstlisting}
\end{frame}

\begin{frame}[fragile]{Syntactic Analysis}
  \begin{lstlisting}
instance Category Syn where
  id  = app0 "id"
  (.) = app2 "."
  \end{lstlisting}
\end{frame}

\begin{frame}[fragile]{Syntactic Analysis}
  \begin{lstlisting}
instance ProductCat Syn where
  exl     = app0 "exl"
  exr     = app0 "exr"
  (&&&)   = app2 "&&&"
  ...
  \end{lstlisting}
\end{frame}

\begin{frame}[fragile]{Syntactic Analysis}
  \begin{lstlisting}
instance TerminalCat Syn where
  it = app0 "it"
  \end{lstlisting}
\end{frame}

\begin{frame}[fragile]{Syntactic Analysis}
  \begin{lstlisting}
instance ClosedCat Syn where
  apply   = app0 "apply"
  curry   = app1 "curry"
  uncurry = app1 "uncurry"
  \end{lstlisting}
\end{frame}

\begin{frame}[fragile]{Syntactic Analysis}
  \begin{lstlisting}
instance BoolCat Syn where
  notC = app0 "not"
  andC = app0 "andC"
  orC  = app0 "orC"
  xorC = app0 "xorC"
  \end{lstlisting}
\end{frame}

\begin{frame}[fragile]{Syntactic Analysis}
  \begin{lstlisting}
instance NumCat Syn a where
  negateC = app0 "negate"
  addC    = app0 "add"
  subC    = app0 "sub"
  mulC    = app0 "mul"
  powIC   = app0 "powI"
  \end{lstlisting}
\end{frame}

\begin{frame}[fragile]{Syntactic Analysis}
  and more code to do with pretty printing, etc.
\end{frame}

\begin{frame}[fragile]{Program Graphs}
  Transforms programs into data-flow graphs, which can be visualised via graphviz.
\end{frame}


\begin{frame}[fragile]{Interval Analysis}
  We used this example in showing the declaration of CCCs.

  Here some example outputs, shown as syntax tree, and in graph form.
\end{frame}

\begin{frame}[fragile]{Interval Analysis}
  \begin{lstlisting}[frame=single]
  instance (Iv a ~ (a :* a), Num a, Ord a) => NumCat IF a where
    ..
    addC = pack (\ ((al,ah),(bl,bh)) -> (al+bl,ah+bh))
    ..
    {-# INLINE addC #-}
    ..
  \end{lstlisting}
\end{frame}

\begin{frame}[fragile]{Interval Analysis}
  \begin{lstlisting}[frame=single]
    runSynME "add" $ toCcc $ ivFun $ uncurry ((+) @Int)
  \end{lstlisting}
  \begin{lstlisting}[frame=single]
     uncurry (curry (apply . (exl &&& exr))) .
     (curry
      (
       (add . (exl . exl &&& exl . exr)
        &&&
        add . (exr . exl &&& exr . exr)
       ) . exr
      ) &&& id
     )
  \end{lstlisting}
\end{frame}

\begin{frame}[fragile]{Interval Analysis}
  \begin{lstlisting}[frame=single]
    runSynCirc "add" $ toCcc $ ivFun $ uncurry ((+) @Int)
  \end{lstlisting}
  \includegraphics{add-iv.pdf}
\end{frame}

\begin{frame}[fragile]{Interval Analysis}
  \begin{lstlisting}[frame=single]
  instance (Iv a ~ (a :* a), Num a, Ord a) => NumCat IF a where
    ..
    subC = addC . second negateC
    ..
    {-# INLINE subC #-}
    ..
  \end{lstlisting}
  \includegraphics{sub-iv.pdf}
\end{frame}

\begin{frame}[fragile]{Interval Analysis}
  \begin{lstlisting}[frame=single]
  instance (Iv a ~ (a :* a), Num a, Ord a) => NumCat IF a where
    mulC = pack (\ ((al,ah),(bl,bh)) ->
               let cs = ((al*bl,al*bh),(ah*bl,ah*bh)) in
                 (min4 cs, max4 cs))
    ..
    {-# INLINE mulC #-}
  \end{lstlisting}
  \begin{center}
    \includegraphics[width=0.8\textwidth]{mul-iv.pdf}
  \end{center}
\end{frame}

\begin{frame}[fragile]{Interval Analysis}
  \begin{lstlisting}[frame=single]
     runSynCirc "thrice-iv" $ toCcc $ ivFun $ \ x -> 3 * x :: Int
  \end{lstlisting}
  \begin{center}
    \includegraphics[width=0.8\textwidth]{thrice-iv.pdf}
  \end{center}
\end{frame}

\begin{frame}[fragile]{Interval Analysis}
  \begin{lstlisting}[frame=single]
    runSynCirc "sqr-iv"    $ toCcc $ ivFun $ sqr @Int
  \end{lstlisting}
  \begin{center}
    \includegraphics[width=1.0\textwidth]{sqr-iv.pdf}
  \end{center}
\end{frame}

\begin{frame}[fragile]{Interval Analysis}
  \begin{lstlisting}[frame=single]
    runSynCirc "magSqr-iv" $ toCcc $ ivFun $ magSqr @Int
  \end{lstlisting}
  \begin{center}
    \includegraphics[width=0.5\textwidth]{magSqr-iv.pdf}
  \end{center}
\end{frame}

\begin{frame}[fragile]{Compiling to Hardware}
  Verilog: a language for integrated circuit design.

  Verilog code can be realised in a CCC. So Haskell programs can be compiled to silicon.
\end{frame}

\begin{frame}[fragile]{Compiling to Hardware}
  \begin{lstlisting}
    runVerilog' "adder" $ \ (x :: Int, y :: Int) -> x + y
  \end{lstlisting}
  \begin{lstlisting}[
    basicstyle=\footnotesize, %or \small or \footnotesize etc.
]
module adder (clk, n0_0_d, n0_1_d, n1_0_q);
  input clk;
  input [31:0] n0_0_d;
  input [31:0] n0_1_d;
  output [31:0] n1_0_q;
  reg [31:0] n0_0;
  reg [31:0] n0_1;
  reg [31:0] n1_0_q;
  always @(posedge clk)
    begin
      n0_0 <= n0_0_d;
      n0_1 <= n0_1_d;
      n1_0_q <= n1_0;
    end
  assign n1_0 = n0_0 + n0_1;
endmodule
  \end{lstlisting}
\end{frame}

\begin{frame}[fragile]{Compiling to Hardware}
  \begin{lstlisting}
    runVerilog' "cond" $ \ (p :: Bool, x :: Int, y :: Int) -> if p then x else y
  \end{lstlisting}
  \begin{lstlisting}[
    basicstyle=\footnotesize, %or \small or \footnotesize etc.
]
module cond (clk, n0_0_d, n0_1_d, n0_2_d, n1_0_q);
  input clk;
  input n0_0_d;
  input [31:0] n0_1_d;
  input [31:0] n0_2_d;
  output [31:0] n1_0_q;
  reg n0_0;
  reg [31:0] n0_1;
  reg [31:0] n0_2;
  reg [31:0] n1_0_q;
  always @(posedge clk)
    begin
      n0_0 <= n0_0_d;
      n0_1 <= n0_1_d;
      n0_2 <= n0_2_d;
      n1_0_q <= n1_0;
    end
  assign n1_0 = n0_0 ? n0_1 : n0_2;
endmodule
  \end{lstlisting}
\end{frame}

\begin{frame}[fragile]{Compiling to Hardware}
  \begin{lstlisting}
    runVerilog' "odd" $ \ (x :: Int) -> x `mod` 2
  \end{lstlisting}
  \begin{lstlisting}[
    basicstyle=\footnotesize, %or \small or \footnotesize etc.
]
module odd (clk, n0_0_d, n2_0_q);
  input clk;
  input [31:0] n0_0_d;
  output [31:0] n2_0_q;
  reg [31:0] n0_0;
  reg [31:0] n2_0_q;
  wire [31:0] n1_0;
  always @(posedge clk)
    begin
      n0_0 <= n0_0_d;
      n2_0_q <= n2_0;
    end
  assign n1_0 = 32'h2;
  assign n2_0 = n0_0 % n1_0;
endmodule
  \end{lstlisting}
\end{frame}

\begin{frame}[fragile]{}
  \frametitle{Linear maps as a category}

  A \textbf{linear map} is a function $f\colon \mathbb{R}^m \rightarrow \mathbb{R}^n$
  such that $f(x + y) = f(x) + f(y)$ and $f(cx) = c\,f(x)$.
  It can also be thought of as an $n \times m$ matrix (where the columns tell you
  what the basis vectors of $\mathbb{R}^m$ map to).

  Linear maps form a category, because:
  \begin{enumerate}
  \item Given $f\colon \mathbb{R}^m \rightarrow \mathbb{R}^n$ and $g\colon \mathbb{R}^n \rightarrow
    \mathbb{R}^p$, we can define the composition $g \circ f \colon \mathbb{R}^m \rightarrow
    \mathbb{R}^p$, which is also a linear map.
  \item Composition of linear maps (alternatively: matrix multiplication) is
    associative.
  \item For any vector space $\mathbb{R}^n$, the identity function
    $1_{\mathbb{R}^n}\colon \mathbb{R}^n \rightarrow \mathbb{R}^n$ is a linear map, and
    has the properties we expect of an identity.
  \end{enumerate}
\end{frame}

\begin{frame}[fragile]
  \frametitle{Types of differentiation}

  \begin{description}
  \item[Symbolic differentiation] Rule-based manipulation of algebraic
    expressions; cumbersome for computers.
  \item[Numeric differentiation] Evaluate the function at two nearby points and
    compute the slope of the resulting line; easy for computers, not so useful
    for humans.
  \item[Automatic differentiation] Tell the computer how to compute the
    derivatives of simple functions, and it will tell you how to compute the
    derivative of any composition of those functions. Easy for a computer,
    useful for humans. But takes more work to set up, and only works for
    functions that are analytically differentiable.
  \end{description}
\end{frame}

\begin{frame}[fragile]
  \frametitle{The chain rule}

  \[(g \circ f)' = (g' \circ f) \cdot (f')\]

  \[\frac{dz}{dx} = \frac{dz}{dy} \frac{dy}{dx}\]

  \[\frac{d\,g(f(x))}{dx} = \frac{d\,g(f(x))}{d\,f(x)} \frac{d\,f(x)}{dx}\]

  In higher dimensions, the derivative of a function is a vector or matrix of
  partial derivatives,
  and the derivative of a composition of two functions is the product of the two
  matrices that give the derivatives of the individual functions.

\end{frame}

\begin{frame}[fragile]
  \frametitle{Derivatives, linear maps, and the chain rule}
  
  The derivative of a function at a point is a \textbf{linear map} (a line,
  plane, linear subspace; equivalently, a matrix giving the slope(s)
  corresponding to a unit move in each dimension). In the category of linear
  maps, composition is \textbf{multiplication} (of matrices, which reduces to
  scalar multiplication for $1 \times 1$ matrices). Differentiation is an operation
  $deriv$ with the property that

  \[
    deriv(g \circ f) = (deriv\, g \circ f) \circ (deriv\, f).
  \]

  where the second $\circ$ on the right-hand side is in the category of linear maps
  (i.e.\ matrix multiplication). {\small (NB: \textbf{not} a functor!)}
  
\end{frame}

\begin{frame}[fragile]
  \frametitle{Differentiation as a functor}

  But we want differentiation to be a functor! The solution is to first map
  every differentiable function $f$ to a pair $(f,f')$ (via $andDeriv$), and
  define composition of such pairs as

  \[
    (g,g') \circ (f,f') = (g \circ f, (g' \circ f) \cdot f').
  \]

  Then $deriv$ is just $snd \circ andDeriv$---and this is obviously a functor.
  
\end{frame}

\begin{frame}[fragile]
  \frametitle{Implementation of automatic differentiation in Haskell}

  \[
    deriv :: (a \rightarrow b) \rightarrow (a \rightarrow (a \multimap_s b))
  \]

  So $deriv$ takes a differentiable function and returns a function that
  associates each input value with a linear map.

  \[
    andDeriv\,f = D (\delprod{f}{deriv\, f})
  \]

  $D$ is a type constructor for function/derivative pairs. As mentioned before,
  $deriv = snd \circ andDeriv$. {\small NB: not quite.}

  Chain rule:
  \[
    Dg \circ Df = D (\lamf{a}{\text{let } \{(b, f') = f\,a; (c,g') = g\,b\} \text{
        in }
    (c, g' \circ f')})
\]

(This is exactly the expression from the previous slide. Confusingly, $f$ and
$g$ now refer to function/derivative pairs, but $f'$ and $g'$ refer to the
derivatives.)
\end{frame}

\section{Possibilities} %% Mark

\begin{frame}[fragile]{Language-to-Language Translation}
  \begin{description}
  \item[CCC] like the Syntax CCC, but constructing code according to the rules of language $X$ (where $X$ is Python, JavaScript, TypeScript, PHP, R, etc.)
  \item[Use] writing type-checked code which can be used in language-specific environments (e.g. JavaScript in browser, R because you need to use an R-only library, PostScript for your printer, etc.)
  \end{description}
\end{frame}

\begin{frame}[fragile]{Probabilistic Modelling}
  \begin{description}
  \item[CCC] each type $a$ replaced by $\Dist{a}$, the distribution over values of type $a$. Function $f$ from $a$ to $b$ replaced by function $f'$ from distributions over $a$ to the resulting distribution over $b$ under the action of $f$.
  \item[Use] take a model mapping independent variables to dependent variables, supply distributions to the independent variables, work out expected distribution of outputs. Calculate $z$-scores (likelihoods) trivially from deterministic models and distributions over dependent inputs.
  \end{description}
\end{frame}

\begin{frame}[fragile]{Correctness Proofs}
  \begin{description}
  \item[CCC] objects are predicates over a possibly-composite value, maps are deductions from predicates over inputs to predicates over outputs
  \item[Use] proving program correctness - each computational step maps onto the deduction about output that it corresponds to
  \end{description}
\end{frame}

\begin{frame}[fragile]{Rigour}
  This approach to compilation extends the mathematical rigour of Haskell (et al) to implementation domains.
\end{frame}

\section{Summary and Conclusion} %% Mark

\begin{frame}[fragile]{}
  C2C offers a mathematically principled way to do program transformation by
  \begin{enumerate}
  \item defining the implementation level as a class of categories (CCCs)
  \item showing how any Haskell program can be mapped onto constructions in those categories
  \item offering some exciting sample translators for programs:
    \begin{itemize}
    \item syntactic trees
    \item data-flow graphs
    \item bound calculation
    \item hardware implementation
    \item linear approximation maps
    \item automatic differentiation
    \item incremental adjustment
    \end{itemize}
    with more possibilities to come
  \item showing the way to mathematically principled remapping of code.
  \end{enumerate}
\end{frame}

\begin{frame}[fragile]{Links}

  \begin{itemize}
  \item \url{http://conal.net/papers/compiling-to-categories/} is the homepage for this project. There you will find links to the paper we've discussed here, slides from Elliott's own talk on this, links to a youtube lecture, and the link to the repository which we drew code/output from.
  \item \url{https://github.com/tyrannomark/CategoryTheory4Haskellions} has the slides for this talk as \texttt{ConCat-talk-20171115.pdf}.
  \end{itemize}
  
  \nocite{Elliott-2017-compiling-to-categories}
  \bibliography{ConCat-talk-20171115}
  \bibliographystyle{abbrv}
\end{frame}

\end{document}
