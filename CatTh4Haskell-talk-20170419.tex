\documentclass[10pt]{beamer}

\usetheme{metropolis}
\usepackage{appendixnumberbeamer}

\usepackage{booktabs}
\usepackage[scale=2]{ccicons}

\usepackage{pgfplots}
\usepgfplotslibrary{dateplot}

\usepackage{xspace}
\newcommand{\MDL}{\textbf{\textsc{mdl}}\xspace}

\usepackage{graphicx}
\usepackage{tikz-cd}

\newcommand{\Cat}[1]{\ensuremath{\underline{\mathbf{#1}}}}
\newcommand{\Obj}[1]{\ensuremath{\mathrm{Obj}(\Cat{#1})}}
\newcommand{\Hom}[3]{\ensuremath{\mathrm{Hom}_{\Cat{#1}}(#2,#3)}}
\newcommand{\Domain}[1]{\ensuremath{\mathbf{Dom}(#1)}}
\newcommand{\Codomain}[1]{\ensuremath{\mathbf{Cod}(#1)}}
\newcommand{\CompCat}[1]{\Cat{\overline{#1}}}
\newcommand{\Comm}[1]{\mathrm{Kom}_{\Cat{#1}}}
\newcommand{\Com}[3]{#3^{#2}}
\newcommand{\SCat}[1]{\Cat{Symms(#1)}}
\newcommand{\FSCat}[1]{\Cat{FSymm(#1)}}
\newcommand{\FkCat}[1]{\Cat{FkSymm(#1)}}
\newcommand{\PkCat}[1]{\Cat{PkSymm(#1)}}
\newcommand{\IrPaths}[1]{\Cat{IrPaths(#1)}}
\newcommand{\FProj}[2]{\ensuremath{W_{\Cat{#1,#2}}}}
\newcommand{\FuncIm}[1]{\mathrm{Im}(#1)}
\newcommand{\term}[1]{\emph{#1}}
\newcommand{\strong}[1]{\textbf{#1}}

\newcommand{\eqnlabel}[1]{\label{eq:#1}}
\newcommand{\eqnref}[1]{(\ref{eq:#1})}
\newcommand{\deflabel}[1]{\label{def:#1}}
\newcommand{\defref}[1]{definition \ref{def:#1}}
\newcommand{\thlabel}[1]{\label{th:#1}}
\newcommand{\thref}[1]{theorem \ref{th:#1}}

% \newcommand{\pullback}{\arrow[d,]}
% \newcommand{\pushout}{\arrow[d,\text{\pigpenfont J}]}

% \newtheorem{threm}{Theorem}[section]
% \newtheorem{lemma}[threm]{Lemma}
\theoremstyle{definition}
% \newtheorem{corollary}[theorem]{Corollary}
% \newtheorem{definition}[theorem]{Definition}
% \newtheorem{example}[theorem]{Example}
\newtheorem{xca}[theorem]{Exercise}

\theoremstyle{remark}
\newtheorem{remark}[theorem]{Remark}

\numberwithin{equation}{section}

%    Absolute value notation
\newcommand{\abs}[1]{\lvert#1\rvert}

%    Blank box placeholder for figures (to avoid requiring any
%    particular graphics capabilities for printing this document).
\newcommand{\blankbox}[2]{%
  \parbox{\columnwidth}{\centering
%    Set fboxsep to 0 so that the actual size of the box will match the
%    given measurements more closely.
    \setlength{\fboxsep}{0pt}%
    \fbox{\raisebox{0pt}[#2]{\hspace{#1}}}%
  }%
}


\title{Category Theory and Haskell}
\subtitle{Parallel Universes}
\date{\today}
\author{Siva Kalyan and T. Mark Ellison}
\institute{Australian National University}
% \titlegraphic{\hfill\includegraphics[height=1.5cm]{logo.pdf}}

\usepackage{listings}
\usepackage{color}
\definecolor{identifierColor}{rgb}{0.65,0.16,0.16}
\lstset{
  frame=none,
  xleftmargin=2pt,
  stepnumber=1,
  numbers=left,
  numbersep=5pt,
  numberstyle=\ttfamily\tiny\color[gray]{0.3},
  belowcaptionskip=\bigskipamount,
  captionpos=b,
  escapeinside={*'}{'*},
  language=haskell,
  tabsize=2,
  emphstyle={\bf},
  commentstyle=\it\color{brown},
  stringstyle=\mdseries\rmfamily\color{green},
  showspaces=false,
  keywordstyle=\bfseries\rmfamily\color{red},
  columns=flexible,
  basicstyle=\small\sffamily\color{blue},
  showstringspaces=false,
  morecomment=[l]\%,
}

\begin{document}

%% SEE https://wiki.haskell.org/wikiupload/8/85/TMR-Issue13.pdf#page=73

\maketitle

\begin{frame}{Table of contents}
  \setbeamertemplate{section in toc}[sections numbered]
  \tableofcontents[hideallsubsections]
\end{frame}


\begin{frame}[fragile]
  \frametitle{Haskell and Category Theory}

  \begin{tabular}{lr}
    \toprule
    Haskell & Category Theory \\
    \midrule
    \textbf{Category} & \textbf{Category} \\
    \textbf{Type} & \textbf{Object} \\
    \textbf{Function} & \textbf{Morphism} \\
    \textbf{\Cat{Hask}} & \textbf{\Cat{Set}} \\
    \textbf{...} & \textbf{Terminal Objects} \\
    \textbf{Value} & \textbf{Global Element} \\
    \textbf{Tuple} & \textbf{Product} \\
    \textbf{Currying, Function Application} & \textbf{Cartesian Closure} \\
    \textbf{Type Constructor, Functor} & \textbf{Functor} \\
    \textbf{...} & \textbf{Natural Transformation} \\
    \textbf{Applicative} & \textbf{...} \\
    \textbf{...} & \textbf{Adjoint Functor Pair} \\
    \textbf{Monad} & \textbf{Monad} \\
    \bottomrule
  \end{tabular}
  

\end{frame}

\section{Categories}

\subsection{Basics}

\begin{frame}[fragile]
  \frametitle{Categories}

  \emph{Categories}

  A \emph{category} $\Cat{C}$ consists of
  \begin{enumerate}
  \item a class $\Obj{C}$ of \emph{objects}, and
  \item for each pair of objects $A,B \in \Obj{C}$, a set $\Hom{C}{A}{B} \in
    \Obj{Set}$ of \emph{arrows} (or \emph{morphisms}) from $A$ to $B$, known as
    a \emph{hom-set}.
  \end{enumerate}\vspace{-1.5\baselineskip}

  \[
    \begin{tikzcd}
      A \arrow[shift left=2]{r}{\Hom{C}{A}{B}} \arrow{r} \arrow[shift right=2]{r} & B
    \end{tikzcd}
  \]

  In Haskell, objects are \emph{types} (\lstinline{Int}, \lstinline{Char}, etc.),
  and arrows are \emph{functions} between types (e.g.\ \lstinline{ord :: Int -> Char}).
\end{frame}

\begin{frame}[fragile]
  \frametitle{Category Laws}
  In a category $\Cat{C}$:
  \begin{enumerate}
  \item Given arrows $f\colon A \rightarrow B$ and $g\colon B \rightarrow C$ in
    $\Cat{C}$, the \emph{composition} $g \circ f \colon A \rightarrow C$ (= \lstinline{g.f}) is also in
    $\Cat{C}$.
  \item Given arrows $f\colon A \rightarrow B$, $g\colon B \rightarrow C$ and $h\colon C \rightarrow D$, $(h \circ
    g) \circ f = h \circ (g \circ f) = h \circ g \circ f$:\vspace{-0.5\baselineskip}
    \[
      \begin{tikzcd}
        A \arrow{r}{f} \arrow[sloped,near end,swap,dashed]{dr}{g \circ f} \arrow[bend
        right=60,sloped,swap,loosely dashed]{drr}{h \circ g \circ f} \arrow[bend
        left=60,sloped,loosely dashed]{drr}{h
          \circ g \circ f} & B \arrow{d}{g}
        \arrow[sloped,near start,dashed]{dr}{h \circ g} & \\
          & C \arrow[swap]{r}{h} & D
      \end{tikzcd}\rlap{.}
    \]
  \item Every object $A \in \Obj{C}$ is associated with an \emph{identity arrow}
    $1_A \colon A \rightarrow A$ (= \lstinline{id}). Given any arrow $f\colon A \rightarrow B$, we have
    \[
      \begin{tikzcd}
        A \arrow[densely dotted]{r}{1_A} \arrow[bend right=30,swap]{rr}{f} & A
        \arrow["f" description]{r}
        \arrow[bend left=30]{rr}{f} & B \arrow[swap,densely dotted]{r}{1_B} & B
      \end{tikzcd}\rlap{.}
    \]
  \end{enumerate}
\end{frame}

\begin{frame}[fragile]
  \frametitle{Examples}

  \begin{center}
    \begin{tabular}{r l l l l}\toprule
    & $\Cat{Set}$ & $\Cat{Hask}$ & $\Cat{POrd}$ & $\Cat{Cat}$ \\\midrule
    \textbf{Objects} & sets & types & items & small cats \\
    \textbf{Morphisms} & functions & functions & $a \leq b$ & functors \\
    \textbf{Composition} & $f \circ g$ & \lstinline!f.g! & transitivity & $F \circ G$ \\
    \textbf{Identity} & $1_A$ & {\lstinline!id!} & $a = a$ & $1_{\Cat{C}}$ \\\bottomrule
  \end{tabular}
  \end{center}

  $\Cat{Hask}$ is \emph{almost} $\Cat{Set}$ ($\Rightarrow$ a type is a ``set of values''),
  but is complicated by the presence of \lstinline{Bottom} ($\bot$) in every type
  (including \lstinline{Void}, which is ``supposed'' to be empty), and the wacky
  behaviour of \lstinline{undefined} with respect to function composition.
\end{frame}

\begin{frame}[fragile]{Category Theory: Terminal Objects}

  \emph{Terminal Objects}

  A \emph{terminal object} is a type $1$ (a.k.a.\ $T$) in $\Obj{C}$, such that there is only a single mapping from any other type $A$ onto that type:

  \[
    \forall A\in \Obj{C}, \left| \Hom{C}{A}{1} \right| = 1.
  \]

  \[
    \begin{tikzcd}[row sep=tiny]
      A \arrow[near start]{dr}{\exists!} & \\
      B \arrow["\exists!" description]{r} & 1\\
      C \arrow[swap,near start]{ur}{\exists!} & 
    \end{tikzcd}
  \]
  
  % Terminal objects in categories:
  % \begin{description}
  %   \item[\Cat{Set}] the singleton set
  %   \item[\Cat{Grp}] the singleton group
  %   \item[\Cat{Top}] the singleton topological space
  %   \item[\Cat{1\downarrow Set}] the singleton pointed set
  %   \item[\Cat{POrd}] the maximum element (if there is one)
  % \end{description}

  In \Cat{Hask}:
  \begin{lstlisting}[frame=single]
    () -- the type corresponding to 1, containing only itself
    terminalMap :: t -> ()
    terminalMap _ = ()
  \end{lstlisting}

\end{frame}

\begin{frame}[fragile,fragile]{Global Elements}
  %% See answer 4 here: http://stackoverflow.com/questions/17380379/where-do-values-fit-in-category-of-hask
  \emph{Global Elements}

  A \emph{global element} of an object $A$ in category $\Cat{C}$ with terminal object $1$ is an arrow $a : 1 \rightarrow A$.
  
  \[
  \begin{tikzcd}
    1 \arrow{r}{a} & A
  \end{tikzcd}
  \]

  In \Cat{Hask}, if we have a value \lstinline{v} in some type \lstinline{A}, we
  can upgrade it to the global element by use of \lstinline{const v}.

  \begin{lstlisting}[frame=single]
    const :: A -> (() -> A)
    const v = \_ -> v
  \end{lstlisting}

\end{frame}

\begin{frame}[fragile]
  \frametitle{Examples}

  \begin{center}
    \begin{tabular}{r l l l l}\toprule
    & $\Cat{Set}$ & $\Cat{Hask}$ & $\Cat{POrd}$ & $\Cat{Cat}$ \\\midrule
    \textbf{Objects} & sets & types & items & small cats \\
    \textbf{Morphisms} & functions & functions & $a \leq b$ & functors \\
    \textbf{Composition} & $f \circ g$ & \lstinline!f.g! & transitivity & $F \circ G$ \\
    \textbf{Identity} & $1_A$ & {\lstinline!id!} & $a = a$ & $1_{\Cat{C}}$ \\
%%    \textbf{Initial obj\rlap{.}} & $\emptyset$ & \texttt{Void} & lwr bnd & $\Cat{0}$ \\
      \textbf{Terminal obj\rlap{.}} & $\{*\}$ & \lstinline!()! & upper bound & $\Cat{1}$ \\\bottomrule
  \end{tabular}
  \end{center}
  
\end{frame}

\begin{frame}[fragile]{Products}

  Given objects $A$, $B$ in $\Obj{C}$ there may be a \emph{(pairwise) product}
  $A\sqcap B \in \Obj{C}$ and \emph{projection arrows} $\pi_A \colon A \sqcap B \rightarrow A$ and $\pi_B
  \colon A \sqcap B \rightarrow B$ such that for any object $X \in \Obj{C}$ and arrows $a \colon X \rightarrow
  A$ and $b \colon X \rightarrow B$ there is a \emph{unique} arrow $x : X \rightarrow A \sqcap B$ such that $a = \pi_A \circ x$ and $b = \pi_B \circ x$:

  \[
  \begin{tikzcd}
    A & A\sqcap B \arrow[swap]{l}{\pi_{A}} \arrow{r}{\pi_{B}} & B \\
    & X \arrow[dotted]{u}{\exists! x} \arrow{ul}{a} \arrow[swap]{ur}{b} & 
  \end{tikzcd}\rlap{.}
  \]

  In other words: Given a particular way of mapping $X$ to $A$ and to $B$, there's only \emph{one} way of mapping $X$ to $A \sqcap B$ such that everything's consistent.

\end{frame}

\begin{frame}[fragile]{Products}

  Alternatively, the triplet $\langle {A \sqcap B, \pi_A, \pi_B} \rangle$ is a \emph{terminal object}
  in the category whose objects are diagrams of the form
  \[
    \begin{tikzcd}
      A & C \arrow{l} \arrow{r} & B
    \end{tikzcd}\rlap{,}
  \]
  and whose arrows are (commutative) diagrams of the form
  \[
    \begin{tikzcd}[row sep=tiny]
       & C \arrow[bend right=15]{dl} \arrow[bend left=15]{dr} & \\
      A & & B \\
       & D \arrow[bend left=15]{ul} \arrow[swap]{uu}{f} \arrow[bend right=15]{ur} & 
    \end{tikzcd}\rlap{.}
  \]

\end{frame}

\begin{frame}[fragile]
  \frametitle{Products in Haskell}
  \begin{lstlisting}[frame=single,mathescape=true]
    (a,b) -- the type containing pairs from types a and b ($A \sqcap B$)
    fst :: (a,b) -> a -- the projection function $\pi_A$
    fst (x,y) = x
    snd :: (a,b) -> b -- the projection function $\pi_B$
    snd (x,y) = y
    mapToA :: c -> a
    mapToB :: c -> b -- any ol' functions
    factorThroughProd :: c -> (a,b)
    factorThroughProd z = (mapToA z, mapToB z)
  \end{lstlisting}

  It should be obvious that\\
  \lstinline{fst.factorThroughProd = mapToA}, and\\
  \lstinline{snd.factorThroughProd = mapToB}.
  
  {\footnotesize (Extra credit: Coproducts. Hint: Disjunctive types.)}
\end{frame}

\begin{frame}[fragile]
  \frametitle{Examples}

  \begin{center}
    \begin{tabular}{r l l l l}\toprule
    & $\Cat{Set}$ & $\Cat{Hask}$ & $\Cat{POrd}$ & $\Cat{Cat}$ \\\midrule
    \textbf{Objects} & sets & types & items & small cats \\
    \textbf{Morphisms} & functions & functions & $a \leq b$ & functors \\
    \textbf{Composition} & $f \circ g$ & \lstinline!f.g! & transitivity & $F \circ G$ \\
    \textbf{Identity} & $1_A$ & {\lstinline!id!} & $a = a$ & $1_{\Cat{C}}$ \\
    % \textbf{Initial obj\rlap{.}} & $\emptyset$ & \lstinline!Void! & lower bound & $\Cat{0}$ \\
    \textbf{Terminal obj\rlap{.}} & $\{*\}$ & \lstinline!()! & upper bound & $\Cat{1}$ \\
    \textbf{Product} & $A \uplus B$ & \lstinline!(a,b)! & $\min(a,b)$ & $\Cat{C} \times \Cat{D}$ \\
    % \textbf{Coproduct} & $A \sqcup B$ & \lstinline!Either a b! & $\max(a,b)$ & $\Cat{C} \sqcup \Cat{B}$ \\\bottomrule
  \end{tabular}
  \end{center}
\end{frame}

\begin{frame}[fragile]
  \frametitle{Exponential Objects}

  Given objects $A$ and $B$ in $\Obj{C}$, an \emph{exponential object} $B^A$
  (also written $[A \rightarrow B]$) is any object such that for any object $C$ and any
  arrow $f\colon C \sqcap A \rightarrow B$,
  \[
  \begin{tikzcd}
    C \sqcap A \arrow[swap,densely dotted]{d}{\exists!} \arrow{dr}{f} & \\
    B^A \sqcap A \arrow[swap]{r}{\mathrm{eval}_B^A} & B
    \end{tikzcd}\rlap{.}
  \]
  
  %% Siva, I think the below might be a bit much for our audience
  
  Alternatively, the pair $\langle {B^A,\mathrm{eval}_B^A} \rangle$ constitutes a terminal
  object in the category whose objects are diagrams of the form
  \[
  \begin{tikzcd}
    C \sqcap A \arrow{r} & B
  \end{tikzcd}\rlap{,}
  \]
  and whose arrows are commutative diagrams of the form
  \[
  \begin{tikzcd}[row sep=tiny]
    D \sqcap A \arrow{dd} \arrow[bend left=15]{dr} & \\
    & B \\
    C \sqcap A \arrow[bend right=15]{ur} &
  \end{tikzcd}\rlap{.}
  \]
\end{frame}

\begin{frame}[fragile,fragile]
  \frametitle{Exponential Objects in Haskell}

  In $\Cat{Hask}$, the exponential object of two types \lstinline{a} and
  \lstinline{b} is the \emph{function type} \lstinline{(a -> b)} (which is
  nothing more than the \emph{hom-set} of \lstinline{a} and \lstinline{b}!). Let's see
  how this satisfies the above definition.

\begin{lstlisting}[frame=single]
eval :: ((a -> b),a) -> b
eval (f,x) = f x
arrow :: (c,a) -> b -- some ol' function
factoredArrow :: (c,a) -> ((a -> b),a)
factoredArrow (y,x) = (\z -> arrow (y,z),x)
\end{lstlisting}
  {\footnotesize{(Spot the currying!)}}

  It should be clear that \lstinline{eval.factoredArrow = arrow} --- and that
  \lstinline{factoredArrow} is the \emph{only} arrow for which this is true.

  A category where we can find the exponential object of any two objects is
  called a \emph{Cartesian closed category} (CCC). $\Cat{Set}$ and $\Cat{Hask}$
  are CCCs.
\end{frame}

\section{Functors}

\begin{frame}[fragile]
  \frametitle{Functors}

  \emph{Functors}

  A \emph{functor} is a mapping $F\colon \Cat{C} \rightarrow \Cat{D}$ that takes objects in
  $\Cat{C}$ to objects in $\Cat{D}$ and arrows in $\Cat{C}$ to arrows in
  $\Cat{D}$, in such a way that
  \begin{enumerate}
  \item for any $A \in \Obj{C}$, $F(1_A) = 1_{F(A)}$:
    \[
      \begin{tikzcd}
        A \arrow[mapsto]{d} \arrow[""' name = id]{r}{1_A} & A \arrow[mapsto]{d} \\
        F(A) \arrow[swap,""' name = Fid]{r}{1_{F(A)}} & F(A) \arrow[mapsto,from=id,to=Fid]
      \end{tikzcd}\rlap{;}
    \]
  \item for any $f\colon A \rightarrow B$ and $g\colon B \rightarrow C$ in $\Cat{C}$, $F(g \circ f) =
    F(g) \circ F(f)$:
    \[
      \begin{tikzcd}[row sep=tiny]
        & B \arrow[mapsto]{dd} \arrow{dr}{g} & \\
        A \arrow[mapsto]{dd} \arrow{ur}{f} \arrow[crossing over,swap,near end]{rr}{g \circ f} & & C \arrow[mapsto]{dd} \\
        & F(B) \arrow{dr}{F(g)} & \\
        F(A) \arrow{ur}{F(f)} \arrow[swap]{rr}{F(g) \circ F(f)} & & F(C)
      \end{tikzcd}\rlap{.}
    \]
  \end{enumerate}
\end{frame}

\begin{frame}[fragile,fragile]
  \frametitle{Functors in Haskell}

  In Haskell, (endo)functors are \emph{type constructors} (e.g.\ the list
  constructor): they take a type (\lstinline{a}) and produce another type (\lstinline{[a]}); and via \lstinline{fmap}, they take an arrow between two
  types (\lstinline{a -> b}) and produce an arrow between the images of those
  two types (\lstinline{[a] -> [b]}).

\begin{lstlisting}[frame=single]
data [] a = [] | a : [a] -- "[]" is the type constructor for lists
fmap f [] = [] -- mapping f over an empty list does nothing
fmap f (x : xs) = (f x) : (fmap f xs)
-- to turn f into a list function, apply f to the head of the list,
-- apply the list version of f to the tail of the list, and construct
\end{lstlisting}

  You can verify that
  \lstinline{fmap id (x : xs) = (id x) : (fmap id xs) = id (x : xs)}, and that
  \lstinline{fmap f (fmap g (x : xs)) = fmap f ((g x) : (fmap g xs))}
  \lstinline{= (f g x) : (fmap f (fmap g xs)) = fmap f g (x : xs)}.
\end{frame}

\begin{frame}[fragile]
  \frametitle{Examples}
  \begin{tabular}{r l l l l}\toprule
    & $\Cat{Set}$ & $\Cat{Hask}$ & $\Cat{POrd}$ & $\Cat{Cat}$ \\\midrule
    \textbf{Objects} & sets & types & items & small cats \\
    \textbf{Morphisms} & functions & functions & $a \leq b$ & functors \\
    \textbf{Composition} & $f \circ g$ & \lstinline!f.g! & transitivity & $F \circ G$ \\
    \textbf{Identity} & $1_A$ & {\lstinline!id!} & $a = a$ & $1_{\Cat{C}}$ \\
    % \textbf{Initial obj\rlap{.}} & $\emptyset$ & \lstinline!Void! & lower bound & $\Cat{0}$ \\
    \textbf{Terminal obj\rlap{.}} & $\{*\}$ & \lstinline!()! & upper bound & $\Cat{1}$ \\
    \textbf{Product} & $A \times B$ & \lstinline!(a,b)! & $\min(a,b)$ & $\Cat{C} \times \Cat{D}$ \\
%%    \textbf{Coproduct} & $A \sqcup B$ & \texttt{A | B} & $\max$ & $\Cat{C} \sqcup \Cat{B}$ \\
    \textbf{Endofunctors} & functors & type const. & \textsc{opt}s & nat.\ trans.\\\bottomrule
  \end{tabular}
\end{frame}

\section{Natural transformations}

\begin{frame}[fragile]
  \frametitle{Natural Transformations}

  \emph{Natural transformations}

  A \emph{natural transformation} $\alpha$ is a mapping between two functors $F\colon
  \Cat{C} \rightarrow \Cat{D}$ and $G\colon \Cat{C} \rightarrow \Cat{D}$. It consists of a family of
  arrows in $\Cat{D}$ (the \emph{components} of $\alpha$) which map each object
  $F(A)$ in the image of $F$ to the corresponding object $G(A)$ in the image of
  $G$. Crucially, the following diagram always commutes:\vspace{-\baselineskip}

  \[
    \begin{tikzcd}
      F(A) \arrow[swap]{d}{\alpha_A} \arrow{r}{F(f)} & F(B) \arrow{d}{\alpha_B} \\
      G(A) \arrow[swap]{r}{G(f)} & G(B)
    \end{tikzcd}\rlap{.}
  \]

  In other words, you can ``jump across'' from $F$ to $G$ at any time; it
  doesn't matter when.
\end{frame}

\begin{frame}[fragile]{Natural Transformations in Haskell}

  A natural transformation in Haskell is given by a function between two type
  constructors. It allows us to ``unwrap'' one constructor and ``repackage'' the
  type with the other. If \lstinline{mu} is a natural transformation, the
  following is necessarily true (cf.\ the commutative diagram):
  \begin{lstlisting}[frame=single]
    mu :: Functor f, Functor g => f a -> g a
    mu . (fmap k) = fmap (mu . k)
  \end{lstlisting}

  According to Milewski, \emph{all} functions with the above type signature are
  natural transformations.

  [Mark will explain the relevance of the following example.]
  
  \begin{lstlisting}[frame=single]
    nu :: Functor g => a -> g a
    nu . k = fmap (nu . k)
  \end{lstlisting}
  
\end{frame}

\begin{frame}[fragile]{Cartesian Closed Categories}

  [(Re)move this slide?]
  
  \emph{Cartesian-Closed Categories (CCC)}
  include
  \begin{description}
    \item[\Cat{Set}] the singleton set, pairs, sets of functions
    \item[\Cat{Hask}] \lstinline{()}, \lstinline{(a,b)}, \lstinline{a -> b}
  \end{description}

\end{frame}

\section{Applicatives}

\begin{frame}[fragile]{Cartesian Closed Categories}

  \emph{Cartesian-Closed Categories (CCC)}

  There is a terminal object $1$.

  There are binary products $\sqcap$.

  There is a bi-functor taking $A \sqcap B$ onto $\Com{C}{A}{B}$, obeying the following rules:
  
  \[
    A \cong 1 \sqcap A \cong \Com{C}{1}{A}
  \]
  
  \begin{equation}
  \Hom{C}{A\sqcap B}{C} \cong \Hom{C}{A}{\Com{C}{B}{C}} \eqnlabel{exp1}
  \end{equation}

  The latter relation is called the \emph{Howard-Curry isomorphism}, or \emph{currying}.
\end{frame}

\begin{frame}[fragile]{Cartesian-Closed Categories}

  [Remove this slide? We don't know any good examples apart from $\Cat{Set}$.]
  
  \emph{Cartesian-Closed Categories (CCC)}
  include
  \begin{description}
    \item[\Cat{Set}] the singleton set, pairs, sets of functions
    \item[\Cat{Grp}] ?
    \item[\Cat{1\downarrow Set}] ?
    \item[\Cat{POrd}] ?
    \item[\Cat{Hask}] \lstinline{()}, \lstinline{(a,b)}, \lstinline{a -> b}
  \end{description}

\end{frame}

\begin{frame}[fragile]{Cartesian-Closed Categories}

  [Eh?! I thought what you've defined here is $\mathrm{eval}_B^C$. Isn't $\lambda$
the name for the map from $\Hom{C}{A\sqcap B}{C}$ to $\Hom{C}{A}{\Com{C}{B}{C}}
\eqnlabel{exp1}$ or something?]
  
  \emph{Cartesian-Closed Categories (CCC)}

  We define the arrow $\lambda$ to be the arrow from $\Com{C}{B}{C}\sqcap B$ to $C$ which corresponds to the identity arrow under the isomorphism in \eqnref{exp1}.

  \[
  \lambda : \Com{C}{B}{C}\sqcap B \rightarrow C \cong 1_{\Com{C}{B}{C}}
  \]

\end{frame}

\begin{frame}[fragile]{Endofunctors in CCCs}

  For any functor $F$, there is a 1-1 correspondence between natural
  transformations from the identity functor to $F$ and the global elements of
  $F(1)$.\footnote{I.e., once you've determined the mapping from $1$ to $F(1)$,
    you've determined the whole behaviour of $F$ --- and vice versa.} Use the
  symbol $i$ to index over the global elements of $F(1)$. Then we have the
  natural transformations $\phi_i$ whose components are:
  
  \[
  \phi_{i,A} : A \rightarrow F A\rlap{,}
  \]
  where
  \[
  f : A \rightarrow B, F(f) \circ \phi_{i,A} = \phi_{i,B} \circ f\rlap{.}
  \]

\end{frame}

\begin{frame}[fragile]{Applicative Functors}

  An \emph{applicative} functor $F$ is a functor $\Cat{C}\rightarrow \Cat{D}$ such that:
  \begin{enumerate}
    \item $\Cat{C}$ is CCC,
    \item $im(F)$ is CCC,
    \item $F$ perserves terminal objects, i.e. $F(1_{\Cat{C}})=1_{\Cat{D}}$,
    \item $F$ perserves products, i.e. $F(A\sqcap B)=F(A)\sqcap F(B)$, and
    \item $F$ perserves the power functor, i.e. $\Com{D}{F A}{(F B)} = F \Com{C}{A}{B}$.
  \end{enumerate}

\end{frame}

\begin{frame}[fragile]{Applicative Functors}

  \begin{lstlisting}[frame=single]
class Functor f => Applicative f where
    -- | Lift a value.
    pure :: a -> f a
    -- | Sequential application.
    (<*>) :: f (a -> b) -> f a -> f b
  \end{lstlisting}

  \lstinline{pure} applies the functor to global elements (arrows from $1$ to $a$).
  \[
  \phi : \Com{Hask}{1}{A} \rightarrow \Com{Hask}{1}{F A}
  \]
  \[
  \phi \circ \epsilon_a \mapsto \epsilon_a'
  \]

  \lstinline{<*>} takes the image of an arrow and of a global element and constructs the image of the composition (also a global element).
  \[
  \psi : (F \Com{Hask}{A}{B})\times\Com{Hask}{1}{F A} \rightarrow \Com{Hask}{1}{F B}
  \]
  \[
  \psi \circ (\epsilon_{f},\epsilon_{a'}) \circ \delta \mapsto \epsilon_b'
  \]

\end{frame}

\begin{frame}[fragile]{Applicatives}

  \emph{The Laws of Applicatives}

  From Wikibooks: \emph{Haskell} chapter on \strong{Applicative Functors}:
  %% https://en.wikibooks.org/wiki/Haskell/Applicative_functors
  
  \begin{enumerate}
  \item \strong{Identity}     \lstinline{pure id <*> v = v}
  \item \strong{Homomorphism} \lstinline{pure f <*> pure x = pure (f x)}
  \item \strong{Interchange}  \lstinline{u <*> pure y = pure ($ y) <*> u}
  \item \strong{Composition}  \lstinline{pure (.) <*> u <*> v <*> w = u <*> (v <*> w)}
  \end{enumerate}

\end{frame}

\begin{frame}[fragile]{Applicatives}

  \emph{The Identity Law}

  \lstinline{pure id <*> v = v}
  \[
  \lambda \circ (\phi \circ H(1_A;1,A^A),H(v;1,F A)) \circ \delta = H(v;1,F A)
  \]

  \[
  \begin{tikzcd}
    1 \arrow{r}{\phi}\arrow{d}[swap]{\epsilon_{1_A},\epsilon_v} & F 1 \arrow{d}{F \epsilon_{1_A},F \epsilon_v} \\
    \Com{C}{A}{A}\times (F A) \arrow{r}{\phi,1} & F (\Com{C}{A}{A}) \times (F A) \arrow{d}{=} \\
    & \Com{C}{F A}{(F A)} \times (F A) \arrow{d}{\lambda_{FA,FA}} \\
    & F A
  \end{tikzcd}
  \]

\end{frame}

\begin{frame}[fragile]{Applicatives}

  \emph{Applicative Functors Example}

  \begin{lstlisting}[frame=single]
instance Applicative Maybe where
    pure x = Just x
    Nothing <*> _ = Nothing
    (Just f) <*> something = fmap f something
  \end{lstlisting}

\end{frame}

\begin{frame}[fragile]{Applicatives}

  \emph{Applicative Functors Example}

  \begin{lstlisting}[frame=single]
instance Applicative [] where
    pure x = [ x ]
    fs <*> xs = [ f x | f <- fs, x <- xs ]
  \end{lstlisting}

\end{frame}


\section{Adjoint Functors}

\begin{frame}[fragile]{Adjoint Functors}

  A 4-tuple $(F,G,\varepsilon,\eta)$ is an \emph{adjunction} between two categories $\Cat{C}$ and $\Cat{D}$
  when:
  \begin{enumerate}
    \item $F$ is a functor from $\Cat{C}\rightarrow\Cat{D}$
    \item $G$ is a functor from $\Cat{D}\rightarrow\Cat{C}$
    \item $\varepsilon$ is a natural transformation from $F\circ G \rightarrow 1_{\Cat{C}}$ (the \emph{counit} of adjunction)
    \item $\eta$ is a natural transformation from $1_{\Cat{D}} \rightarrow G\circ F$ (the \emph{unit} of adjunction)
    \item $(\varepsilon F)\circ(F \eta) = 1_{F}$
    \item $(G \varepsilon)\circ(\eta G) = 1_{G}$
  \end{enumerate}

  Another way of talking about adjoints is that $(F,G)$ composed with $Hom$ form an adjoint pair
  when there is a natural isomorphism
  \[
  Hom(F-,-) \cong_{\Phi} Hom(-,G-)
  \]
  
\end{frame}

\begin{frame}[fragile]{Adjoint Functors}
  
\end{frame}

\begin{frame}[fragile]
  \frametitle{Examples}
  \begin{tabular}{r l r l}\toprule
    \Cat{C} & $\Cat{D}$ & $F$ & $G$ \\
    \hline
    \Cat{Set} & \Cat{Set^2} & $\Delta$ & $\sqcap$ \\
    \Cat{Set^2} & \Cat{Set} & $\sqcup$  & $\Delta$ \\
    \Cat{Set} & \Cat{Grp} & Free & Forgetful \\
    \Cat{Set} & \Cat{Top} & Discrete & Forgetful \\
    \Cat{Top} & \Cat{Set} & Forgetful & Trivial
  \end{tabular}
\end{frame}

\begin{frame}[fragile]{Adjoint Functors}

  Although Haskell doesn't talk about adjoints as much as Monads, one adjunction-pair is fundamental to FP: the Currying adjunction.
  \[
  \Hom{C}{X\times Y}{Z} \cong \Hom{C}{X}{Z^Y}
  \]
  Think of product-with-$Y$ and to-the-power-of-$Y$ as functors, then product is the left-adjoint of power.
  
\end{frame}

\begin{frame}[fragile]{Adjoint Functors}
  
\end{frame}

\section{Monads}

\begin{frame}[fragile]{Monads}

  A \emph{monad} is a triple $(T,\eta,\mu)$ where
  \begin{enumerate}
  \item $T: \Cat{C}\to \Cat{C}$ is an endofunctor,
  \item $\eta: 1_{\Cat{C}} \to T$ is the counit of adjunction
  \item $\mu: T \circ T \rightarrow T$ is the unit of adjunction
  \end{enumerate}

  \[
  \begin{tikzcd}
    \mathrm{M1} & T^3 \arrow{d}[swap]{\mu T}\arrow{r}{T\mu}  & T^2 \arrow{d}[swap]{\mu} &
    \mathrm{M2} & T \arrow{d}[swap]{T\eta}\arrow{r}{\eta T}\arrow[dash,double]{dr} & T^2 \arrow{d}{\mu} \\
    & T^2 \arrow{r}[swap]{\mu} & T & & T^2 \arrow{r}[swap]{\mu} & T
  \end{tikzcd}
  \]

  {\small Wikipedia, Monad\_(category\_theory)}
  
\end{frame}

\begin{frame}[fragile]{Monads}

  Earlier definition of monad.
  
  \begin{lstlisting}[frame=single]
    class Functor m => Monad m where
      return      :: a -> m a
      join        :: m (m a) -> m a
  \end{lstlisting}

  {\small King \& Wadler 1993}

  This is how it is defined now.
  
  \begin{lstlisting}[frame=single]
    class Applicative m => Monad m where
    (>>=)       :: forall a b. m a -> (a -> m b) -> m b
    return      :: a -> m a
  \end{lstlisting}

  {\small GHC.Base base-4.9.1.0}

  \begin{lstlisting}[frame=single]
    (>>=) m g = join (fmap g m)
  \end{lstlisting}

\end{frame}

\begin{frame}[fragile]{Monad Laws}

  (omitting two that guarantee functorhood of \lstinline{f})
  
  \begin{lstlisting}[frame=single]
    fmap (f . return) = return . f -- return is a nat trans
    fmap (f . join) = join . $ fmap (fmap f) -- join is a nat trans
    join . (fmap join) = join . join -- diagram M1
    join . return = id -- bottom-left of M2
    join . (fmap return) = id -- top-right of M2
  \end{lstlisting}

  {\small King \& Wadler 1993}
  
\end{frame}

\begin{frame}[fragile]{Other Connections}
  \begin{description}
  \item[Arrows] 
  \item[Comonads] 
  \item[Lens] 
  \item[Kleisli Arrows] 
  \end{description}
\end{frame}

\section{Further Reading}

\begin{frame}[fragile]{Further Reading}

  ADD: Milewski, Walters
  
  \nocite{elkins_calculating_2009}
  \nocite{diel:_blog}
  \bibliography{CatTh4Haskell-talk-20170419}
  \bibliographystyle{abbrv}

\end{frame}

\end{document}
